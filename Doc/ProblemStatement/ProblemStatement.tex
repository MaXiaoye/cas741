\documentclass{article}

\usepackage{tabularx}
\usepackage{booktabs}

\title{CAS 741: Problem Statement\\Breaking Effect}

\author{Xiaoye Ma max58}

\date{}

%% Comments

\usepackage{color}

\newif\ifcomments\commentstrue

\ifcomments
\newcommand{\authornote}[3]{\textcolor{#1}{[#3 ---#2]}}
\newcommand{\todo}[1]{\textcolor{red}{[TODO: #1]}}
\else
\newcommand{\authornote}[3]{}
\newcommand{\todo}[1]{}
\fi

\newcommand{\wss}[1]{\authornote{blue}{SS}{#1}}
\newcommand{\an}[1]{\authornote{magenta}{Author}{#1}}


\begin{document}

\maketitle

\begin{table}[hp]
\caption{Revision History} \label{TblRevisionHistory}
\begin{tabularx}{\textwidth}{llX}
\toprule
\textbf{Date} & \textbf{Developer(s)} & \textbf{Change}\\
\midrule
2017-09-13 & Xiaoye Ma & Version0a, new document\\
2017-09-14 & Xiaoye Ma & Version0b, more description in the first question to make the problem clearer.\\
Date2 & Name(s) & Description of changes\\
... & ... & ...\\
\bottomrule
\end{tabularx}
\end{table}

What problem are you trying to solve?\\

This project tries to implement breaking effect in codes for 3-D models in video games without help from any plug-in. Including different shapes 3-D objects breaking based on physics and pieces interacting with the momentum provided by the breaking force. The breaking effect program simulates 3-D objects’ destruction process in vision by implementing scientific computing functions. This project concentrates on calculation while HCI or GUI are not important parts. Applied force is decided in codes in advance as input and pieces movement is the output after calculation.\\      

Why is this an important problem?\\

Breaking effect is widely used in 3-D video games and animations for both nature phenomenons and destruction made by human. It is meaningful in game development especially for act games with a background of wars or battles. As one of the most significant measurements on visualization level, nowadays game developers pay more attention to visual effect in large-scale games. A good performance in breaking effect makes a game more realistic and contributes a lot to players’ game experiences.\\ 

What is the context of the problem you are solving?\\

The breaking effect can be mainly used by game developers in 3-D video games for improving visualization performance, to be played when an 3-D object is broken or an internal explosion happens . It can also be used in movies and 3-D animations by any editor properly. This project does not pay much attention on model creation part so that the program needs to rely on existing 3-D models in unity3D. However the mathematics computing method can be imported for any available usage. \\


\end{document}