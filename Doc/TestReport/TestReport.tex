\documentclass[12pt, titlepage]{article}

\usepackage{booktabs}
\usepackage{tabularx}
\usepackage{hyperref}
\usepackage{xr}
\externaldocument{../SRS/SRS}
\externaldocument{../Design/MG/MG} 
\hypersetup{
	colorlinks,
	citecolor=black,
	filecolor=black,
	linkcolor=red,
	urlcolor=blue
}
\usepackage[round]{natbib}
\newcommand{\rref}[1]{R\ref{#1}}
\newcommand{\ddref}[1]{DD\ref{#1}}
\newcommand{\mref}[1]{M\ref{#1}}
%% Comments

\usepackage{color}

\newif\ifcomments\commentstrue

\ifcomments
\newcommand{\authornote}[3]{\textcolor{#1}{[#3 ---#2]}}
\newcommand{\todo}[1]{\textcolor{red}{[TODO: #1]}}
\else
\newcommand{\authornote}[3]{}
\newcommand{\todo}[1]{}
\fi

\newcommand{\wss}[1]{\authornote{blue}{SS}{#1}}
\newcommand{\an}[1]{\authornote{magenta}{Author}{#1}}


\begin{document}

\title{Test Report: Breaking Effect} 
\author{Marshall Xiaoye Ma}
\date{\today}
	
\maketitle

\pagenumbering{roman}

\section{Revision History}

\begin{tabularx}{\textwidth}{p{3cm}p{2cm}X}
\toprule {\bf Date} & {\bf Version} & {\bf Notes}\\
\midrule
2017-12-08 & 1.0 & New document\\
\bottomrule
\end{tabularx}

~\newpage

\section{Symbols, Abbreviations and Acronyms}

See SRS Documentation at \url{https://github.com/MaXiaoye/cas741/blob/master/Doc/SRS/SRS.pdf}

\newpage

\tableofcontents

\listoftables %if appropriate

\listoffigures %if appropriate

\newpage

\pagenumbering{arabic}

This document provides a report of the results of the testing performed on the Breaking Effect application. Both system testing and unit testing are covered. Traceability between testing and both requirements and modules is given in the final section.
The tests that were implemented in this report follow the Test Plan document. Please refer to this document for further information at \url{https://github.com/MaXiaoye/cas741/blob/master/Doc/TestPlan/TestPlan.pdf}.

\section{Functional Requirements Evaluation}

\subsection{Getting input from user}
\label{Sec_TestInput}

This test suite is designed to ensure the program can handle both valid and invalid inputs from user that cover \rref{R_Inputs} and \rref{R_VerifyOutput}. Inputs provided by user include target object, explosion level (also known as initial momentum after explosion) and coefficient of friction on the ground.
\paragraph{Assumption}
\begin{itemize}
	\item In all test cases, $(X,Y,Z) = (a,b,c)$ means a target object with position $(a,b,c)$. The target object for all test cases named "targetObj" consists of 32 cubes with size 1 x 1 x 1. It locates under projectDir/Assets/prefab/targetObj 
	
	\item $\approx$ consider equal with a delta of 0.01
	
	\item All test codes are in dir /src/Assets/script/BETest.cs
\end{itemize}

\begin{enumerate}
	
	\item{testInput}
	
	Type: Functional, Dynamic, Automated
	
	Initial State: New session
	
	Input: Please refer to table below
	
	Expect output: Please refer to table below.
	
	How test will be performed: Automated unit test
	
	Result: Pass
\end{enumerate}

\begin{center}
	\begin{tabular}{p{2cm} p{5cm} p{4cm} p{4cm}}
		\hline
		\textbf{Name} & \textbf{In} & \textbf{Expect output} & \textbf{Result} \\
		\hline
		CorrectIn &$E = 5$, $(X,Y,Z) = (0,0,0)$, $\mu_{k} = 0.05$ & - &Pass\\
		NoMu & $E = 5$, $(X,Y,Z) = (0,0,0)$  & InvalidMuException & Pass\\
		NoObj & $E = 5$, $\mu_{k} = 0.05$ & NoObjException & Pass\\
		NoExpLv &$(X,Y,Z) = (0,0,0)$, $\mu_{k} = 0.05$ & NoELvException & Pass\\
		NonNumE &$E = a$, $(X,Y,Z) = (0,0,0)$, $\mu_{k} = 0.05$ & InvalidELvException & Pass\\
		NonNumMu &$E = 5$, $(X,Y,Z) = (0,0,0)$, $\mu_{k} = a$ & InvalidMuException & Pass\\
		OutScoE & $E=-1$ or $E=11$,$(X,Y,Z) = (0,0,0)$,$\mu_{k} = 0.05$  & InvalidELvException & Pass\\
		OutScoC & $E = 5$, $(X,Y,Z) = (0,99999,0)$, $\mu_{k} = 0.05$ & InvalidCoorYException & Pass\\
		OutScoM & $E = 5$, $(X,Y,Z) = (0,0,0)$, $\mu_{k} = 99$ &InvalidMuException & Pass\\
		\hline		
	\end{tabular}
\end{center}

\subsection{Acquire pieces}
\label{Sec_testGravityCenter}

	This section covers \rref{R_ClassPiece}, \rref{R_Piece} in SRS that class PieceObj is defined as required. Instance of PieceObj can be created by acquiring each piece in the scene correctly.There is an assumption for this section that all tests in \ref{Sec_TestInput} are passed.\\

\paragraph{Acquire all pieces}

\begin{enumerate}
	
	\item{pieceInit}
	
	Type: Functional, Dynamic, Automated
	
	Initial State: New Session.
	
	Input: $E = 5$, $(X,Y,Z) = (0,0,0)$, $\mu_{k} = 0.2$
	
	Expect output: A list of piece objects: pieceObj that pieceObj.length = 32. Piece shape is random that is decided by cutting method.
	
	How test will be performed: Automated unit test.	
	
	Result: Pass
	
\end{enumerate}

\paragraph{Calculation of angle between initial speed and horizontal($\theta _{1}$) as well as the angle between x axiom and projection on horizontal of initial speed($\theta _{2}$)}
\begin{enumerate}
	\item{testAngles1\\}
	
	Type: Functional, Dynamic, Automated
	
	Initial State: New Session.
	
	Input: $E = 5$, $(X,Y,Z) = (0,0,0)$, $\mu_{k} = 0.2$, $(x_{n},y_{n},z_{n}) = (1.5, 2.0, 0.5)$;
	
	Expect output: $pieceObj[0].\theta_{1}=arctan \frac{y_{n} - Y}{\sqrt{(z_{n}-Z)^2+(x_{n}-X)^2}} \approx 0.90$ and $pieceObj[0].\theta_{2}=arctan \frac{x_{n}-X}{z_{n}-Z} \approx 1.25$
		
	How test will be performed: Automated unit test. 
	
	Result: Pass
	\item{testAngles2\\}
	
	Type: Functional, Dynamic, Automated
	
	Initial State: New Session.
	
	Input: $E = 5$, $(X,Y,Z) = (0,0,0)$, $\mu_{k} = 0.2$, $(1.5, 2.0, 1.5)$
	
	Expect output: $\theta_{1}\approx 0.76$ and $\theta_{2}\approx 0.79$
	
	How test will be performed: Automated unit test.
	
	Result: Pass
	
	\item{testAngles3\\}
	
	Type: Functional, Dynamic, Automated
	
	Initial State: New Session.
	
	Input: Input: $E = 5$, $(X,Y,Z) = (0,0,0)$, $\mu_{k} = 0.2$, $(0,y_{n},0)$ and $0y_{n}$ can't be zero
	
	Expect output: $\theta_{1}=90^{\circ} \approx 1.57$ and $\theta_{2}=90^{\circ} \approx 1.57$
	
	How test will be performed: Automated unit test.
	
	Result: Pass
	
\end{enumerate}

\subsection{Calculation of displacement for each piece}
\label{Sec_TestMotionAir}

This section covers \rref{R_Output1} that displacement of each piece in the air should be calculated correctly. Testing for edge cases and error handling will be covered in Section \ref{Sec_UnitTest}.\\ 
There is an assumption for this section that all tests in \ref{Sec_TestInput}, \ref{Sec_testGravityCenter} are passed.
\begin{enumerate}
	
	\item{testMotionAir\\}
	
	Type: Functional, Dynamic, Automated
	
	Initial State: New Session.
	
	Input: $E = 5$, $(X,Y,Z) = (0,0,0)$, $\mu_{k} = 0.2$, $\Delta t = 0.02$, $t = 2.86$
	
	Expect output: $S_{x}=v_{0}\cdot cos\theta _{1}\cdot cos\theta _{2}\cdot \Delta t \approx 0.59$, $S_{z}=v_{0}\cdot cos\theta _{1}\cdot sin\theta _{2}\cdot \Delta t \approx 0.20$ and $S_{y}=(v_{0}\cdot sin\theta _{1} - gt)\cdot \Delta t-\frac{1}{2}g \cdot \Delta t^{2} \approx 0.40$. Above output should be shown both in text in console and in vision on the screen.
	
	How test will be performed: Automated unit test.
	
	Result: Pass
	
	\item{testMotionGround\\}
	
	Type: Functional, Dynamic, Automated
	
	Initial State: New Session.
	
	Input: $E = 5$, $(X,Y,Z) = (0,0,0)$, $\mu_{k} = 0.2$, $\Delta t = 0.02$, $t = 2.86$
	
	Expect Output: $a=\mu_{k}g$, $v_{x} = v_{x}^{'} + a\Delta t$, $v_{z} = v_{z}^{'} + a\Delta t$, $S_{x}=v_{x}\cdot \Delta t-\frac{1}{2}a\Delta t^{2} \approx 0.58$, $S_{z}=v_{z}\cdot \Delta t-\frac{1}{2}a\Delta t^{2} \approx 0.19$ 
	
	How test will be performed: Automated unit test.
	
	Result: Pass
\end{enumerate}

\section{Nonfunctional Requirements Evaluation}

\subsection{Usability}

	Installation of Unity3D is not in the scope of Breaking Effect's usability test.

\begin{enumerate}
	
	\item{Participants\\}
	Classmate: Halona (Yuzhi)	
	
	\item{Document given\\}
	Readme.doc under project folder.
	
	\item{Task for participants\\}
	Run test case successfully by reading Readme.doc individually . Without help from
	others.
	
	\item{How test will be measured and performed\\}
	According to content of Readme.doc and complexity of the program. If all participants can run test case successfully by only reading Readme.doc individually, then Usability of the program is good.
	
	\item{Result\\} Halona didn't use Unity3D before and does not have Unity installed on laptop. She used my laptop and spent 19 minutes to successfully run the demo by importing package and run BreakingEffect on a new object created by herself. During the process I still need to provide some help so I improved the user guide according to problems she met: 1. If all items in package should be imported. 
	2. Where can I find the scripts and objects after importing package.
	3. Is there any requirement for my own object.
\end{enumerate}


\subsection{Performance test}

\begin{enumerate}
	
	\item{testPerformance\\}
	
	Type: Nonfunctional,Dynamic, Manual
	
	Initial State: New session
	
	Input/Condition: $E = 5$, $(X,Y,Z) = (0,0,0)$, $\mu_{k} = 0.2$ The range of the number of pieces is 4 - 128.
	
	Output/Result: Motion of each piece.
	
	How test will be performed: Adjust amount of pieces to see the performance that if program can run smoothly in huge amount of pieces.
	
	Result: Frame per second(FPS) keeps 60 when number of pieces increases from 4 to 128. So there is no extra performance issue by using BreakingEffect. Performance depends on capability of device and number of objects that a Unity3D scene already has.
\end{enumerate}

\subsection{Maintainability}
	
	\begin{enumerate}
		
		\item{testMaintain}
		
		Type: Nonfunctional,Dynamic, Manual
		
		Initial State: New session
		
		How test will be performed: Extend the program with additional physics effect rebound (not perfectly).
		
		Result: Extend BreakingEffect with rebounding: Need to modify Piece Object Module, Collision with ground detection Module. \an{Don't have enough to implement this perfectly}
		
	\end{enumerate}
	
\subsection{Reusability}
	
	\begin{enumerate}
		
		\item{testReuse}
		
		Type: Nonfunctional,Dynamic, Manual
		
		Initial State: New session
		
		How test will be performed: Combine Breaking Effect with existing Unity3D project.
		
		Result: Combine BreakingEffect with official Unity3D demo project "Standard Assets Example Project". It can be used on it by following User Guide.
	\end{enumerate}

\section{Unit Testing}
\label{Sec_UnitTest}
Unit Test Generator provided by Visual Studio 2017 will be used to implement automated unit testing for this project.

\subsection{Unit test for Piece Object Module (\mref{mPO})}
\label{Sec_UmPO}

\begin{enumerate}	
	\item{ConstructorPieceObj}
	
	Type: Functional, Dynamic, Automated
	
	Initial State: New session.
	
	Input: Refer to table below;
	
	Output: Refer to table below
	
	How test will be performed: Automated unit testing.
\end{enumerate}

\begin{center}
	\begin{tabular}{p{2.5cm} p{4cm} p{6cm} p{4cm}}
		\hline
		\textbf{Name} & \textbf{In} & \textbf{Expect Output} & \textbf{Result} \\
		\hline
		UTPieceClass1 &$E = 5$, $(X,Y,Z) = (0,0,0)$, $\mu_{k} = 0.2$ &pieceObj[0].obj.name="Cube1"&Pass\\
		UTPieceClass2 &$E = 5$, $(X,Y,Z) = (0,0,0)$, $\mu_{k} = 0.2$ &pieceObj[0].onGround = false&Pass\\
		UTPieceClass3 &$E = 5$, $(X,Y,Z) = (0,0,0)$, $\mu_{k} = 0.2$ &pieceObj[0].initSpeed = 50&Pass\\
		UTPieceClass4 &$E = 5$, $(X,Y,Z) = (0,0,0)$, $\mu_{k} = 0.2$ &pieceObj[0].gravity = -9.8&Pass\\
		UTPieceClass5 &$E = 5$, $(X,Y,Z) = (0,0,0)$, $\mu_{k} = 0.2$ &pieceObj[0].stop = false&Pass\\
		UTPieceClass6 &$E = 5$, $(X,Y,Z) = (0,0,0)$, $\mu_{k} = 0.2$ &pieceObj[0].speedLastFrameX = 1&Pass\\
		UTPieceClass7 &$E = 5$, $(X,Y,Z) = (0,0,0)$, $\mu_{k} = 0.2$ &pieceObj[0].speedLastFrameZ = 1&Pass\\
		\hline		
	\end{tabular}
\end{center}

\subsection{Unit test for Target Object Module (\mref{mTO})}
\label{Sec_UmTO}
\begin{enumerate}
	
	\item{UTTarObj1\\}
	
	Type: Functional, Dynamic, Automated
	
	Initial State: New session.
	
	Input: $E = 5$, $(X,Y,Z) = (0,0,0)$, $\mu_{k} = 0.2$
	
	Expect output: targetObj.transform.childCount = 32
	
	How test will be performed: Automated unit testing.
	
	Result: Pass
\end{enumerate}

\subsection{Unit test for Collision with ground detection Module (\mref{mOC})}
\label{Sec_UmOC}
\begin{enumerate}
	
	\item{UTCollision1\\}
	
	Type: Functional, Dynamic, Manually
	
	Initial State: New session.
	
	Input: $E = 5$, $(X,Y,Z) = (0,0,0)$, $\mu_{k} = 0.2$
	
	Expect Output: It is clear to see that pieces drop on the ground.(Don't go through the ground). pieceObj[1].onGround = true
	
	How test will be performed: Manually test by developer
	
	Result: Pass
\end{enumerate}

\subsection{Unit test for Output Module (\mref{mOM})}
\label{Sec_UmOM}
\begin{enumerate}
	
	\item{UTOutput1\\}
	
	Type: Functional, Dynamic, Automated
	
	Initial State: New session.
	
	Input: $E = 5$, $(X,Y,Z) = (0,0,0)$, $\mu_{k} = 0.05$
	
	Expect output: Visualization on screen works well.
	
	How test will be performed: Manual test by developer
	
	Result: Pass
\end{enumerate}

\subsection{Unit test for Camera controlling Module (\mref{mCC})}
\label{Sec_UmCC}
\begin{enumerate}
	
	\item{UTOutput1\\}
	
	Type: Functional, Dynamic, Automated
	
	Initial State: New session.
	
	Input: $E = 5$, $(X,Y,Z) = (0,0,0)$, $\mu_{k} = 0.05$
	
	Expect output: User can control camera moving by keyboard and mouse.
	
	How test will be performed: Manual test by developer
	
	Result: Pass
\end{enumerate}


\section{Changes Due to Testing}

\begin{itemize}	
	\item Put codes run every frame from Update() function to FixedUpdate() function
	\item  Add function GetCenter() and GetExplosionPoint() to get accurate explosion point that is bottom center of target center instead of position attribute defined by Unity3D.
	\item Add some functions to get static value due to testing requirement.
	\item Add codes to check if speed of a piece object reaches zero on the ground. If it reaches zero then stop the object immediately to prevent the speed increases in opposite direction.
	\item Add camera controlling module to provide free camera controlling when program running to improve experience and make testing easier.
	\item Separate collision detecting module to a new file and improve it so that BreakingEffect support multiple target objects.  
\end{itemize}	
	
\section{Automated Testing}
	Automated Testing are mentioned in test cases with Automated type in previous sections.
\section{Trace to Requirements}

\newpage
\begin{table}[h!]
	\centering
	\begin{tabular}{|c|c|c|c|c|c|}
		\hline
		& \rref{R_Inputs} & \rref{R_VerifyOutput} & \rref{R_ClassPiece}& \rref{R_Piece} &\rref{R_Output1}\\
		\hline
		\ref{Sec_TestInput}&X &X & & & \\ \hline
		\ref{Sec_testGravityCenter} & & &X &X &\\ \hline
		\ref{Sec_TestMotionAir} & & & & &X\\ \hline
	\end{tabular}
	\caption{Traceability Matrix Showing the Connections Between Test Cases and Requirements}
	\label{Table:inputTest2}
\end{table}
		
\section{Trace to Modules}		

\begin{table}[h!]
	\centering
	\begin{tabular}{|c|c|c|c|c|c|c|c|c|}
		\hline
		& \mref{mIF} & \mref{mPO} & \mref{mOGC}& \mref{mDC1} &\mref{mTO} &\mref{mOC}&\mref{mOM}&\mref{mCC}\\
		\hline
		\ref{Sec_TestInput}&X & & & & & & &\\ \hline
		\ref{Sec_testGravityCenter} & & &X & & & & &\\ \hline
		\ref{Sec_TestMotionAir} & & & &X & & & &\\ \hline
		\ref{Sec_UmPO} & &X & & & & & &\\ \hline
		\ref{Sec_UmTO} & & & & &X & & &\\ \hline
		\ref{Sec_UmOC} & & & & & &X & &\\ \hline
		\ref{Sec_UmOM} & & & & & & &X &\\ \hline
		\ref{Sec_UmCC} & & & & & & & &X\\ \hline
	\end{tabular}
	\caption{Traceability Matrix Showing the Connections Between Test Cases and Modules}
	\label{Table:inputTest3}
\end{table}

\section{Code Coverage Metrics}

\bibliographystyle{plainnat}

\bibliography{SRS}

\end{document}
