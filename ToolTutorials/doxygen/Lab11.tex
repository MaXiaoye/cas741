\documentclass[12pt,fleqn]{article}
\usepackage{url}
\usepackage{hyperref}
\hypersetup{colorlinks=true,
    linkcolor=blue,
    citecolor=blue,
    filecolor=blue,
    urlcolor=blue,
    unicode=false}
\setlength {\topmargin} {-.15in}
\setlength {\textheight} {8.6in}
\newcommand{\be}{\begin{enumerate}}
\newcommand{\ee}{\end{enumerate}}
\newcommand{\bi}{\begin{itemize}}
\newcommand{\ei}{\end{itemize}}
\newcommand{\bc}{\begin{center}}
\newcommand{\ec}{\end{center}}
\newcommand{\bsp}{\begin{sloppypar}}
\newcommand{\esp}{\end{sloppypar}}
\renewcommand{\labelenumii}{\theenumii.}

\begin{document}
\bc
{\Large \textbf{SOFTWARE ENGINEERING 3XA3}}\\[2mm]
{\large \textbf{Software Engineering Practice \& Experience: Software Project
Management }}\\[6mm]
{\large \textbf{Dr.~Spencer Smith}}\\[2mm]
{\large \textbf{McMaster University, Fall 2016}}\\[6mm]
{\LARGE \textbf{Laboratory 11 Doxygen}}\\[4mm]
{\large Revised: September 6, 2016}
\ec
\medskip
\noindent
This lab will introduce Doxygen. However, for your projects, you are free to
select any automatic
documentation software that you prefer. Doxygen is a documentation generator
that
we will emphasize in the lab, but you are not constrained to use it going
forward.
\subsection*{Components of Lab}
\be
\item Introduction to Doxygen
\item Doxygen Exercises
\ee
\subsection*{Details}
%Assume students have no knowledge of Doxygen.
%BEGIN INTRO SCRIPT
%
% What is Doxygen? 
%
% A tool for generating documentation from the code source files.
% It is a documentation system written mainly for C (also supports other
% popular languages) similar to JavaDoc.

% What does it do? 
%
% 1. Generating documents in LaTeX, HTML, PostScript, etc.
% 2. Extracting code structure, visualizing relationships between the elements
% using graphs and diagrams.
% 3. Creating normal documentation.

% Students should check to see if Doxygen supports their programming language.
% However, it supports languages such as C,C+%+,Java, Python that
% majority of the groups are using.

% Comment blocks format for C-like languages:
% Block format: /**
%		      *
%		      *
%		      */

% Comment blocks for Python:(double up the comment marker (#) on the first line)
% Block format: ##
%		        #
% Using this comment block style, they are able to use Doxygen tags.

% For Python, they can also use Python documentation string syntax ( @package
% docstring) and """ as comment marker. In this case Doxygen is able to extract
% documenttion but Doxygn tags cannot be used. First style ( # and Doxygen 
% tags) is preferred.

% Doxygen tags students must be familiar with:
%		@mainpage, @file, @brief, @date, @version,
% 		@details, @param, @var,@code, @endcode
% 		@note, @warning, @bug, @todo, @author

% Doxygen writes a Makefile into the latex directory. Hence, to get the pdf ,
% students can use make in the latex directory in Linux/OSX or
% run make.bat in the latex output folder in Windows. The generated file with
% the default configuration is refman.pdf.

% Since python looks more like Java than like C, students need to
% optimize output for Java in the config file.
% If Doxygen is not extracting method descriptions, make sure that EXTRACT_ALL
% is set to yes.

%END INTRO SCRIPT

Please review the following materials on Doxygen:
\bi
\item An introduction to Doxygen : \url{http://www.stack.nl/~dimitri/doxygen/}
\item You can download Doxygen from:
\\
 \url{http://www.stack.nl/~dimitri/doxygen/download.html}
\item Tutorials on Doxygen: 
	\bi
	\item \url{https://www.youtube.com/watch?v=qYTg_YOiPNA}
	\item \url{https://www.youtube.com/watch?v=px4PTEFwioU}
	\ei

\item In general, writing Doxygen comment blocks involves three main steps:
	\bi
\item Writing comments at the beginning of each file containing file name,
description, author, date, version, known bugs, etc.
\item Writing comment blocks to document functions/methods/classes. All functions/methods/classes must
have a comment block containing a brief description, list of the parameters,
author, date, return values, etc.
\item Writing comments for variables.
	\ei
	
Read about Doxygen comment block styles for your chosen programming language
and follow the examples at:
\\
 \url{http://www.stack.nl/~dimitri/doxygen/manual/docblocks.html}
\\
 An example for C-like languages:
\\
\url{https://www.icts.uiowa.edu/confluence/display/ICTSit/Doxygen+Examples}
\\
\ei
 
\subsection*{Exercises}
\bi
\item Select one of these files: \href{https://gitlab.cas.mcmaster.ca/smiths/se3xa3/blob/master/Labs/L07/Box3D.java}{\texttt{Box3D.Java}}, \href{https://gitlab.cas.mcmaster.ca/smiths/se3xa3/blob/master/Labs/L07/Box3D.py}{\texttt{Box3D.py}} and put comments in the code such
that Doxygen incorporates them in the documentation it generates.
\ei
\subsection*{For The TAs (Students can ignore this section)}
Before the conclusion of the lab, please make sure that you have tested each
student
to verify that they understand the basics of the lab exercise.  For students
that understand the basics, please give them a grade of 1 for this Lab
Exercise.
\end{document}