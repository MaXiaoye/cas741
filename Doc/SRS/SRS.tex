\documentclass[12pt]{article}

\usepackage{amsmath, mathtools}
\usepackage{amsfonts}
\usepackage{amssymb}
\usepackage{graphicx}
\usepackage{colortbl}
\usepackage{xr}
\usepackage{hyperref}
\usepackage{longtable}
\usepackage{xfrac}
\usepackage{tabularx}
\usepackage{float}
\usepackage{siunitx}
\usepackage{booktabs}
\usepackage{caption}
\usepackage{pdflscape}
\usepackage{afterpage}

\usepackage[round]{natbib}

%\usepackage{refcheck}

\hypersetup{
    bookmarks=true,         % show bookmarks bar?
      colorlinks=true,       % false: boxed links; true: colored links
    linkcolor=red,          % color of internal links (change box color with linkbordercolor)
    citecolor=green,        % color of links to bibliography
    filecolor=magenta,      % color of file links
    urlcolor=cyan           % color of external links
}

\input{../Comments}

% For easy change of table widths
\newcommand{\colZwidth}{1.0\textwidth}
\newcommand{\colAwidth}{0.13\textwidth}
\newcommand{\colBwidth}{0.82\textwidth}
\newcommand{\colCwidth}{0.1\textwidth}
\newcommand{\colDwidth}{0.05\textwidth}
\newcommand{\colEwidth}{0.8\textwidth}
\newcommand{\colFwidth}{0.17\textwidth}
\newcommand{\colGwidth}{0.5\textwidth}
\newcommand{\colHwidth}{0.28\textwidth}

% Used so that cross-references have a meaningful prefix
\newcounter{defnum} %Definition Number
\newcommand{\dthedefnum}{GD\thedefnum}
\newcommand{\dref}[1]{GD\ref{#1}}
\newcounter{datadefnum} %Datadefinition Number
\newcommand{\ddthedatadefnum}{DD\thedatadefnum}
\newcommand{\ddref}[1]{DD\ref{#1}}
\newcounter{theorynum} %Theory Number
\newcommand{\tthetheorynum}{T\thetheorynum}
\newcommand{\tref}[1]{T\ref{#1}}
\newcounter{tablenum} %Table Number
\newcommand{\tbthetablenum}{T\thetablenum}
\newcommand{\tbref}[1]{TB\ref{#1}}
\newcounter{assumpnum} %Assumption Number
\newcommand{\atheassumpnum}{P\theassumpnum}
\newcommand{\aref}[1]{A\ref{#1}}
\newcounter{goalnum} %Goal Number
\newcommand{\gthegoalnum}{P\thegoalnum}
\newcommand{\gsref}[1]{GS\ref{#1}}
\newcounter{instnum} %Instance Number
\newcommand{\itheinstnum}{IM\theinstnum}
\newcommand{\iref}[1]{IM\ref{#1}}
\newcounter{reqnum} %Requirement Number
\newcommand{\rthereqnum}{P\thereqnum}
\newcommand{\rref}[1]{R\ref{#1}}
\newcounter{lcnum} %Likely change number
\newcommand{\lthelcnum}{LC\thelcnum}
\newcommand{\lcref}[1]{LC\ref{#1}}

\newcommand{\progname}{Breaking Effect} % PUT YOUR PROGRAM NAME HERE

\usepackage{fullpage}

\begin{document}

\title{Breaking Effect} 
\author{Xiaoye Ma, max58@mcmaster.ca}
\date{\today}
	
\maketitle

~\newpage

\pagenumbering{roman}

\section{Revision History}

\begin{tabularx}{\textwidth}{p{3cm}p{2cm}X}
\toprule {\bf Date} & {\bf Version} & {\bf Notes}\\
\midrule
2017-10-01 & 1.0 & New document\\
\bottomrule
\end{tabularx}

~\newpage

\section{Reference Material}

This section records information for easy reference.

\subsection{Table of Units}

Throughout this document SI (Syst\`{e}me International d'Unit\'{e}s) is employed
as the unit system.  In addition to the basic units, several derived units are
used as described below.  For each unit, the symbol is given followed by a
description of the unit and the SI name.
~\newline

\renewcommand{\arraystretch}{1.2}
%\begin{table}[ht]
  \noindent \begin{tabular}{l l l} 
    \toprule		
    \textbf{symbol} & \textbf{unit} & \textbf{SI}\\
    \midrule 
    \si{\metre} & length & metre\\
    \si{\kilogram} & mass	& kilogram\\
    \si{\second} & time & second\\
    \si{\joule} & energy & Joule\\
    \bottomrule
  \end{tabular}
  %	\caption{Provide a caption}
%\end{table}

\subsection{Table of Symbols}

The table that follows summarizes the symbols used in this document along with
their units.  The choice of symbols was made to be consistent with the heat
transfer literature and with existing documentation for solar water heating
systems.  The symbols are listed in alphabetical order.

\renewcommand{\arraystretch}{1.2}
%\noindent \begin{tabularx}{1.0\textwidth}{l l X}
\noindent \begin{longtable*}{l l p{12cm}} \toprule
\textbf{symbol} & \textbf{unit} & \textbf{description}\\
\midrule 
$S$ & unit of length & displacement
\\
$t$ & \si[per-mode=symbol] s & the time in seconds since the start of the game
\\  
$v_{0}$ & \si[per-mode=symbol] m/s & initial speed
\\ 
$a$ & $m/s^2$ & acceleration
\\
$g$ & $m/s^2$ & gravity acceleration
\\
$\Delta E_{k}$ & J & variation of kinetic energy
\\
$\Delta E_{p}$ & J & variation of potential energy
\\
$W_{f}$ & J & work done by kinetic friction
\\
$x_{n}$ & m & x coordinates of gravity center of piece n \wss{Explain the
  meaning of $n$?  Also, you should format it like a mathematical variable ($n$,
not n).}
\\
$y_{n}$ & m & y coordinates of gravity center of piece n
\\
$z_{n}$ & m & z coordinates of gravity center of piece n
\\
$\theta_{1}$ & degree & angle between initial speed and horizontal
\\
$\theta_{2}$ & degree & angle between
x axiom and projection on horizontal of initial speed
\\
$S_{x}$ & m & displacement on direction of x axiom
\\
$S_{y}$ & m & displacement on direction of y axiom
\\
$S_{z}$ & m & displacement on direction of z axiom
\\
$\mu_k$ &  & coefficient of friction
\\
\bottomrule
\end{longtable*}

\subsection{Abbreviations and Acronyms}

\renewcommand{\arraystretch}{1.2}
\begin{tabular}{l l} 
  \toprule		
  \textbf{symbol} & \textbf{description}\\
  \midrule 
  A & Assumption\\
  DD & Data Definition\\
  GD & General Definition\\
  GS & Goal Statement\\
  IM & Instance Model\\
  LC & Likely Change\\
  PS & Physical System Description\\
  R & Requirement\\
  SRS & Software Requirements Specification\\
  \progname{} & Breaking Effect\\
  T & Theoretical Model\\
  \bottomrule
\end{tabular}\\

\newpage

\tableofcontents

~\newpage

\pagenumbering{arabic}

\section{Introduction}

Because of the development of video games industrial and hardware such as CPU
and GPU, there is a higher demand for high level experience in game
visualization. Breaking effect plays a more important role in the visualization
level of large scale video games. \wss{Tell the reader what the breaking effect
  is.}  This project simulates the process of 3D objects’ destruction.

The following section provides an overview of the Software Requirements Specification (SRS) for a breaking effect program. The developed program will be referred to as Breaking Effect (BE). This section explains the purpose of this document, the scope of the system, the characteristics of the intended readers and the organization of the document.

\wss{The text is better for version control, and for reading in other editors,
  if you use a hard-wrap at 80 characters}

\subsection{Purpose of Document}

The main purpose of this document is to describe the modeling of breaking effect. The goals and theoretical models used in the breaking effect code are provided, with an emphasis on explicitly identifying assumptions and unambiguous definitions. 

\subsection{Scope of Requirements} 

The scope of the requirements is limited to breaking effect of a single 3D object applied by force. Interact force between objects and collision among several objects are not in the scope. Given the appropriate inputs, the code for BE is intended to calculate pieces motion and display the process of target 3D object breaking in vision. The project is implemented in a game engine. 3-D objects and piece generation function are provided by platform. 

\subsection{Characteristics of Intended Reader} 

Reviewers of this documentation should have a basic knowledge in physics motion
theory and an understanding of differential equations.  \wss{For the
  characteristics of intended reader try to be more specific about the
  education.  What degree?  What course areas?  What level?}

\subsection{Organization of Document}

The organization of this document follows the template for an SRS for scientific
computing software proposed by Smith and Lai (2005); Smith et al. (2007). The
presentation follows the standard pattern of presenting goals, theories,
definitions, and assumptions. \wss{You should use BibTeX for this.  The original
  template showed you how to do this.  If you have questions about how to use
  BibTeX, please ask your classmates, or me.}

\section{General System Description}

This section identifies the interfaces between the system and its environment,
describes the user characteristics and lists the system constraints.

\subsection{System Context}

\begin{figure}
	\centering
	\includegraphics[width=0.7\linewidth]{./f18}
	\caption{}
	\label{fig:f18}
\end{figure}

\begin{itemize}
\item User Responsibilities:
\begin{itemize}
\item Provide inputs and make sure they are in an appropriate range \wss{Try to
    be more specific about what inputs the user provides.  Determining whether
    inputs are in the appropriate range seems like something your program could
    easily test.}
\end{itemize}
\item \progname{} Responsibilities:
\begin{itemize}
\item Detect data type mismatch, such as a string of characters instead of a
  floating point number
\item Determine if the inputs satisfy the required physical and software
  constraints.​ \wss{Is the appropriate range (from the user responsibilities
    above) included in the constraints?}
\item Calculate the required outputs
\end{itemize}
\end{itemize}

\subsection{User Characteristics} \label{SecUserCharacteristics}

The end user of \progname{} should have a basic understanding of Physics and 3D
models. \wss{Please be more specific on the users education background and level.}

\subsection{System Constraints}

There are no system constraints.  \wss{I think you have to include Unity as a
  system constraint, or remove the mention of it elsewhere in the document.  You
  have obviously (for practical reasons) decided that you are going to use
  Unity.  You should explicitly inform the user of this.  I also suggest that
  you mention that there are other options for physics engines, but for
  practical reasons they will not be considered in this project.}

\section{Specific System Description}

This section first presents the problem description, which gives a high-level
view of the problem to be solved.  This is followed by the solution characteristics
specification, which presents the assumptions, theories, definitions and finally
the instance models. 

\subsection{Problem Description} \label{Sec_pd}

This project tries to implement running time breaking effect in codes for 3-D models in unity3D without help from any similar plug-in. Including different shapes 3-D objects breaking based on physics and pieces interacting with the momentum provided by the breaking force. The breaking effect program simulates 3-D objects’ destruction process in vision by implementing scientific computing functions. This project concentrates on calculation while HCI or GUI are not important parts. Applied force is decided in codes in advance as input and trace of motion is the output after calculation.

\subsubsection{Terminology and  Definitions}

This subsection provides a list of terms that are used in the subsequent
sections and their meaning, with the purpose of reducing ambiguity and making it
easier to correctly understand the requirements:

\begin{itemize}

\item Focus point: The location where explosion happeds \wss{spell check!}
\item initial momentum level: Initial momentum of pieces  

\end{itemize}

\subsubsection{Physical System Description}

The physical system of \progname{} includes visual terrian \wss{spell check!} and a target 3-D object which will break into pieces.

% \begin{figure}[h!]
% \begin{center}
% %\rotatebox{-90}
% {
%  \includegraphics[width=0.5\textwidth]{<FigureName>}
% }
% \caption{\label{<Label>} <Caption>}
% \end{center}
% \end{figure}

\subsubsection{Goal Statements}

\noindent Given the target object, coefficient of friction, focus point and initial momentum level, the goal statements are:

\begin{itemize}

\item[GS\refstepcounter{goalnum}\thegoalnum \label{G_speed}:] Calculate initial status including pieces generation and initial speed of each piece.​

\item[GS\refstepcounter{goalnum}\thegoalnum \label{G_motion}:] Calculate trace of motion for each piece.

\end{itemize}

\subsection{Solution Characteristics Specification}

The instance models that govern \progname{} are presented in
Subsection~\ref{sec_instance}.  The information to understand the meaning of the
instance models and their derivation is also presented, so that the instance
models can be verified.

\subsubsection{Assumptions}

This section simplifies the original problem and helps in developing the
theoretical model by filling in the missing information for the physical
system. The numbers given in the square brackets refer to the theoretical model
[T], general definition [GD], data definition [DD], instance model [IM], or
likely change [LC], in which the respective assumption is used.

\begin{itemize}

\item[A\refstepcounter{assumpnum}\theassumpnum \label{A_mechanical}:]
 The forms of energy that are relevant for this problem are kinetic energy and
 potential energy. Thermal energy is assumed to be negligible in the air. All
 other forms of energy are negligible. \wss{but you have friction?  Work will be
   done that eventually uses up the initial energy.}

\item[A\refstepcounter{assumpnum}\theassumpnum \label{A_airFriction}:]
Air friction will be ignored.

\item[A\refstepcounter{assumpnum}\theassumpnum \label{A_powder}:]
Powders and airborn \wss{spell check!} will not be considered. Because airborn
has random shapes, motions and they are seriously influenced by air friction
\wss{missing period.} \wss{What makes a piece powder?  How is powder defined?}

\item[A\refstepcounter{assumpnum}\theassumpnum \label{A_initialPE}:]
Initial kinetic energy and potential energy of target object are zero. \wss{You
  say in your goal statement that there is an initial momentum.  If the kinetic
  energy is zero, then the momentum will always be }

\item[A\refstepcounter{assumpnum}\theassumpnum \label{A_ground}:]
The same coefficient of friction everywhere on flat ground.​

\item[A\refstepcounter{assumpnum}\theassumpnum \label{A_piece}:]
Pieces are not decomposable. 

\item[A\refstepcounter{assumpnum}\theassumpnum \label{A_collision}:]
Collision between pieces will not be considered. 

\end{itemize}

\subsubsection{Theoretical Models}\label{sec_theoretical}

This section focuses on the general equations and laws that \progname{} is based
on. 

~\newline

\noindent
\begin{minipage}{\textwidth}
	\renewcommand*{\arraystretch}{1.5}
	\begin{tabular}{| p{\colAwidth} | p{\colBwidth}|}
		\hline
		\rowcolor[gray]{0.9}
		Number& T\refstepcounter{theorynum}\thetheorynum \label{T_CME}\\
		\hline
		Label&\bf Conservation of mechanical energy\\
		\hline
		Equation&  $E_{mech}=E_{k}+E_{p}$\\
		\hline
		
		Description & 
		The total mechanical energy (defined as the sum of its potential
                              and kinetic energies) of a particle being acted on
                              by only conservative forces is constant. The
                              potential energy, $E_{p}$, depends on the position
                              of an object subjected to a conservative
                              force. The kinetic energy, $E_{k}$, depends on the
                              speed of an object and is the ability of a moving
                              object to do work on other objects when it
                              collides with them.  \wss{What about the work done
          by nonconservative forces, like friction?  This should be part of
                              your conservation of energy.  The equation you
                              have written is only true in the case where the
                              work done by nonconservative forces is zero.  You
                              should also reference the assumptions that are
                              necessary for your theoretical model to apply.}\\
		\hline
		Source &
		\url{http://www.nuclear-power.net/laws-of-conservation/law-of-conservation-of-energy/conservation-of-mechanical-energy/}\\
		% The above web link should be replaced with a proper citation to a publication
		\hline
		Ref.\ By & \iref{IM_DIA}\\
		\hline
\end{tabular}
\end{minipage}\\

~\newline

\noindent
\begin{minipage}{\textwidth}
	\renewcommand*{\arraystretch}{1.5}
	\begin{tabular}{| p{\colAwidth} | p{\colBwidth}|}
		\hline
		\rowcolor[gray]{0.9}
		Number& T\refstepcounter{theorynum}\thetheorynum \label{T_WKF}\\
		\hline
		Label&\bf Work done by kinetic friction\\
		\hline
		Equation&  $\Delta E_{k}=W_{f}$\\
		\hline
		
		Description & 
		Kinetic energy loses due to kinetic energy. $W_{f}$ is work done by kinetic friction. As a result, kinetic energy transform to internal energy.\\
		\hline
		Source &
		\url{http://teacher.pas.rochester.edu/phy121/lecturenotes/Chapter07/Chapter7.html}\\
		% The above web link should be replaced with a proper citation to a publication
		\hline
		Ref.\ By & \iref{IM_DOG}\\
		\hline
	\end{tabular}
\end{minipage}\\

~\newline

\subsubsection{General Definitions}\label{sec_gendef}

There is no general definition for current problem.

\subsubsection{Data Definitions}\label{sec_datadef}

This section collects and defines all the data needed to build the instance
models. The dimension of each quantity is also given. 

~\newline

\noindent
\begin{minipage}{\textwidth}
\renewcommand*{\arraystretch}{1.5}
\begin{tabular}{| p{\colAwidth} | p{\colBwidth}|}
\hline
\rowcolor[gray]{0.9}
Number& DD\refstepcounter{datadefnum}\thedatadefnum \label{DD_UAM}\\
\hline
Label& \bf Displacement in uniformly accelerated motion\\
\hline
Symbol &$S$\\
\hline
% Units& $Mt^{-3}$\\
% \hline
  SI Units & M\\
  \hline
  Equation&$S= v_{0}t+\frac{1}{2}at^{2}$\\
  \hline
  Description & 
                 The above equation gives us the distance traveled without
                having to know the final velocity of the object. Where $t$ is
                time duration and $v_{0}$ is initial speed. Acceleration $a$ is
                defined as the rate of change of velocity with respect to time,
                in a given direction. This would mean that if an object has an
                acceleration of 1 ms-2 \wss{You can write this as m/s$^2$} it will increase its velocity (in a given
                direction) 1 ms-1 every second that it accelerates. This
                equation is tenable under \aref{A_mechanical},
                \aref{A_airFriction} and \aref{A_ground} in this project.
                \wss{This equation only applies for constant acceleration.  As
                the object loses energy the acceleration will decrease.  None of
                your assumptions mention assuming constant acceleration.  The
                equation you are using comes from kinematics, where you don't
                need to worry about forces.  I think you do have to worry about
                forces, which moves it to the area of kinetics.}
  \\
  \hline
  Sources&~\url{http://ibphysicsstuff.wikidot.com/uniformaccmotion}  \\
  \hline
  Ref.\ By & \iref{IM_DIA},\iref{IM_DOG}\\
  \hline
\end{tabular}
\end{minipage}\\

\subsubsection*{Derivation of how to derive the equation from relationship of position, velocity and acceleration}

~\newline
We assume $v_{0}$ is initial velocity and $v_{t}$ is the final velocity. 

~\newline
Then we have:
$v_{t}=v_{0}+at$

~\newline
We can get average velocity $v_{a}$
~\newline
$v_{a}=\frac{v_{0}+v_{t}}{2}$

~\newline
So we can get displacement

~\newline 
$S=v_{a}t=\frac{v_{0}+v_{t}}{2}t$

~\newline
$S=\frac{v_{0}+v_{0}+at}{2}t=v_{0}t+\frac{1}{2}at^{2}$


~\newline

\noindent
\begin{minipage}{\textwidth}
	\renewcommand*{\arraystretch}{1.5}
	\begin{tabular}{| p{\colAwidth} | p{\colBwidth}|}
		\hline
		\rowcolor[gray]{0.9}
		Number& DD\refstepcounter{datadefnum}\thedatadefnum \label{DD_Fk}\\
		\hline
		Label& \bf Kinetic friction\\
		\hline
		Symbol &$F_{k}$\\
		\hline
		% Units& $Mt^{-3}$\\
		% \hline
		SI Units & N\\
		\hline
		Equation&$F_{k}=\mu_{k}F_{n}=\mu_{k}mg$\\
		\hline
		Description & 
		Kinetic friction $F_{k}$ is a force that acts between moving
                              surfaces. An object that is being moved over a
                              surface will experience a force in the opposite
                              direction as its movement. The magnitude of the
                              force depends on the coefficient of kinetic
                              friction between the two kinds of material. Every
                              combination is different. The coefficient of
                              kinetic friction is assigned the Greek letter "mu"
                              (μ), with a subscript "k".  \wss{Use ``quote'' to
                              get correct quotation marks} The force of kinetic
                              friction is μk times the normal force on an
                              object, and is expressed in units of Newtons (N).
                              In this project, $F_{n}$ equals to
                              gravity. \wss{gravity is an acceleration, not a
                              force.  Your force is $mg$, where $m$ is the mass
                              of the object.  Do you know the mass of the objects?}
		\\
		\hline
		Sources&~\url{http://www.softschools.com/formulas/physics/kinetic_friction_formula/92/}  \\
		\hline
		Ref.\ By & \iref{IM_DOG}\\
		\hline
	\end{tabular}
\end{minipage}\\

\subsubsection{Instance Models} \label{sec_instance}    

This section transforms the problem defined in Section~\ref{Sec_pd} into 
one which is expressed in mathematical terms. It uses concrete symbols defined 
in Section~\ref{sec_datadef} to replace the abstract symbols in the models 
identified in Sections~\ref{sec_theoretical} and~\ref{sec_gendef}.

The goals \gsref{G_speed} and \gsref{G_motion} are solved by \iref{IM_angle} to \iref{IM_DOG}.
Piece generation is done by calling cutting function in game engine.   

~\newline

%Instance Model 1

\noindent
\begin{minipage}{\textwidth}
	\renewcommand*{\arraystretch}{1.5}
	\begin{tabular}{| p{\colAwidth} | p{\colBwidth}|}
		\hline
		\rowcolor[gray]{0.9}
		Number& IM\refstepcounter{instnum}\theinstnum \label{IM_angle}\\
		\hline
		Label& \bf Find the angle between $v_{0}$ and horizontal. Find the angle between x axiom and projection on horizontal of initial speed \\
		\hline
		Input&$x_{n}$,$y_{n}$,$z_{n}$\\
		\hline
		Output&$\theta_{1}=arctan \frac{|z_{n}|}{\sqrt{x_{n}^2+y_{n}^2}}$\\
		&$\theta_{2}=arctan \frac{y_{n}}{x_{n}}$\\
		\hline
		Description&$x_{n}$ is x coordinates of gravity center of piece n.\\
		&$y_{n}$ is y coordinates of gravity center of piece n.\\
		&$z_{n}$ is z coordinates of gravity center of piece n.\\
		\hline
		Sources&~\ \ \\
		\hline
		Ref.\ By & \iref{IM_DIA}\\
		\hline
	\end{tabular}
\end{minipage}\\

~\newline

\noindent
\begin{minipage}{\textwidth}
\renewcommand*{\arraystretch}{1.5}
\begin{tabular}{| p{\colAwidth} | p{\colBwidth}|}
  \hline
  \rowcolor[gray]{0.9}
  Number& IM\refstepcounter{instnum}\theinstnum \label{IM_DIA}\\
  \hline
  Label& \bf Uniformly accelerated motion to find displacement in the air\\
  \hline
  Input&$v_{0}$,$\theta_{1}$ from IM1,$\theta_{2}$ from IM1,$t$,$g$\\
  \hline
  Output&$S_{x}=v_{0}\cdot cos\theta _{1}\cdot cos\theta _{2}\cdot t$\\
  &$S_{y}=v_{0}\cdot cos\theta _{1}\cdot sin\theta _{2}\cdot t$,\\
  &$S_{z}=v_{0}\cdot sin\theta _{1}\cdot t-\frac{1}{2}gt^{2}$\\
  \hline
  Description&$v_{0}$ is the initial speed.\\
  &$t$ is time from beginning.\\
  &$\theta _{1}$ is the angle between $v_{0}$ and horizontal.\\
  &$\theta _{2}$ is the angle between x axiom and projection on horizontal of initial speed.\\
  & The above equation applies as long as the piece moving in the air.
  \\
  \hline
  Sources&~\ \ \\
  \hline
  Ref.\ By & \\
  \hline
\end{tabular}
\end{minipage}\\

~\newline

\noindent
\begin{minipage}{\textwidth}
	\renewcommand*{\arraystretch}{1.5}
	\begin{tabular}{| p{\colAwidth} | p{\colBwidth}|}
		\hline
		\rowcolor[gray]{0.9}
		Number& IM\refstepcounter{instnum}\theinstnum \label{IM_DOG}\\
		\hline
		Label& \bf Uniformly accelerated motion to find displacement on the ground\\
		\hline
		Input&$v_{0}$,$\theta_{1}$,$\theta_{2}$,$t$,$g$,$\mu_{k}$\\
		\hline
		Output&$a=\mu_{k}g$\\
		&$S_{x}=v_{0}\cdot cos\theta _{1}\cdot cos\theta _{2}\cdot t-\frac{1}{2}at^{2}$\\
		&$S_{y}=v_{0}\cdot cos\theta _{1}\cdot sin\theta _{2}\cdot t-\frac{1}{2}at^{2}$,\\
		\hline
		Description&$v_{0}$ is the initial speed.\\
		&$t$ is time from beginning.\\
		&$\theta _{1}$ is the angle between $v_{0}$ and horizontal.\\
		&$\theta _{2}$ is the angle between x axiom and projection on horizontal of initial speed.\\
		&$\mu_{k}$ is the Coefficient of friction.\\
		& The above equation applies as long as the piece moving on the ground.
		\\
		\hline
		Sources&~\ \ \\
		\hline
		Ref.\ By & \\
		\hline
	\end{tabular}
\end{minipage}\\

%~\newline

\subsubsection*{Derivation of how to get angle between $v_{0}$ and horizontal}

We have initial location (X,Y,Z) of target object, and location $(x_{n},y_{n},z_{n})$ for piece n, then we can calculate the angle between $v_{0}$ and horizontal by:
~\newline
\\
$tan\theta_{1}=\frac{|z_{n}-Z|}{\sqrt{(x_{n}-X)^{2}+(y_{n}-Y)^{2}}}$
~\newline
\\
If location of target object is (0,0,0), then we have:
~\newline
\\
$tan\theta_{1}=\frac{|z_{n}|}{\sqrt{x_{n}^{2}+y_{n}^{2}}}$
~\newline
\\
$\theta_{1}=arctan \frac{|z_{n}|}{\sqrt{x_{n}^2+y_{n}^2}}$
~\newline
\\

\wss{You really need to work on the physics in your project. You might
  understand the physics better when you proceed to the next steps.  For
  instance, you will learn about the problems with applying DD1 - your particles
  would never stop, they would just keep accelerating.  For your project faking
  a rational design process should be very helpful.}
 
\wss{You don't say how the object will be broken into pieces.  You have this
  goal, but as far as I can tell, you never return to it.  You should add goals
  to your traceability matrix, so that you would see this omission.}

\subsubsection{Data Constraints} \label{sec_DataConstraints}    

Tables~\ref{TblInputVar} and \ref{TblOutputVar} show the data constraints on the
input and output variables, respectively.  The column for physical constraints gives
the physical limitations on the range of values that can be taken by the
variable.  The column for software constraints restricts the range of inputs to
reasonable values.  The constraints are conservative, to give the user of the
model the flexibility to experiment with unusual situations.  The column of
typical values is intended to provide a feel for a common scenario.  The
uncertainty column provides an estimate of the confidence with which the
physical quantities can be measured.  This information would be part of the
input if one were performing an uncertainty quantification exercise.

The specification parameters in Table~\ref{TblInputVar} are listed in
Table~\ref{TblSpecParams}.

\begin{table}[!h]
  \caption{Input Variables} \label{TblInputVar}
  \renewcommand{\arraystretch}{1.2}
\noindent \begin{longtable*}{l l l l c} 
  \toprule
  \textbf{Var} & \textbf{Physical Constraints} & \textbf{Software Constraints} &
                             \textbf{Typical Value} & \textbf{Uncertainty}\\
  \midrule 
  $X$ &  & $L_{\text{min}} \leq X \leq L_{\text{max}}$ & 0 & 10\%
  \\
  $Y$ &  & $L_{\text{min}} \leq Y \leq L_{\text{max}}$ & 0 & 10\%
  \\
  $Z$ &  & $L_{\text{min}} \leq Z \leq L_{\text{max}}$ & 0 & 10\%
  \\
  $x_{n}$ &  & $L_{\text{min}} \leq x_{n} \leq L_{\text{max}}$ & 0 & 10\%
  \\
  $y_{n}$ &  & $L_{\text{min}} \leq y_{n} \leq L_{\text{max}}$ & 0 & 10\%
  \\
  $z_{n}$ &  & $L_{\text{min}} \leq z_{n} \leq L_{\text{max}}$ & 0 & 10\%
  \\
  $v_{0}$ & $v_{0} > 0$ & $v_{\text{min}} \leq v_{0} \leq v_{\text{max}}$ & 20 m/s & 10\%
  \\
  $\mu_{k}$ & $\mu_{k} > 0$ & $\mu_{\text{min}} \leq \mu_{k} \leq \mu_{\text{max}}$ & 0.05 & 10\%
  \\
  \bottomrule
\end{longtable*}
\end{table}

\noindent 

\begin{table}[!h]
\caption{Specification Parameter Values} \label{TblSpecParams}
\renewcommand{\arraystretch}{1.2}
\noindent \begin{longtable*}{l l} 
  \toprule
  \textbf{Var} & \textbf{Value} \\
  \midrule 
  $v_\text{min}$ & 10 m/s\\
  $v_\text{max}$ & 100 m/s\\
  $L_\text{min}$ & -1000\\
  $L_\text{max}$ & 1000 m/s\\
  $\mu_\text{min}$ & 0\\
  $\mu_\text{max}$ & 1\\
  \bottomrule
\end{longtable*}
\end{table}

\begin{table}[!h]
\caption{Output Variables} \label{TblOutputVar}
\renewcommand{\arraystretch}{1.2}
\noindent \begin{longtable*}{l l} 
  \toprule
  \textbf{Var} & \textbf{Physical Constraints} \\
  \midrule 
  $S_{x}$ & $S_{x} \geq 0$
  \\
  $S_{y}$ & $S_{y} \geq 0$
  \\
  $S_{z}$ & $S_{z} \geq 0$
  \\
  \bottomrule
\end{longtable*}
\end{table}

\subsubsection{Properties of a Correct Solution} \label{sec_CorrectSolution}

\noindent
A correct solution must exhibit the principle of motion as well as conservation of energy.

\section{Requirements}

This section provides the functional requirements, the business tasks that the
software is expected to complete, and the nonfunctional requirements, the
qualities that the software is expected to exhibit.

\subsection{Functional Requirements}

\noindent \begin{itemize}

\item[R\refstepcounter{reqnum}\thereqnum \label{R_Inputs}:] 
	Input the following quantities, which define the tank parameters, material properties
	and initial conditions:
	\noindent \begin{longtable*}{l l l} 
		\toprule
		\textbf{symbol} & \textbf{unit} & \textbf{description}\\
		\midrule 
		$X$ & & x coordinates of target object
		\\
		$Y$ & & y coordinates of target object
		\\
		$Z$ & & z coordinates of target object
		\\
		$E$ & & Initial momentum level
		\\
		$\mu_{k}$ & & Coefficient of friction everywhere on flat ground
		\\
		\bottomrule
	\end{longtable*}

\item[R\refstepcounter{reqnum}\thereqnum \label{R_OutputInputs}:] Use inputs in R1 to find initial speed $v_{0}$, as follows:

$v_{0}=E \cdot 10$

\item[R\refstepcounter{reqnum}\thereqnum \label{R_VerifyOutput}:] Verify that the inputs satisfy the required physical constraints shown in Table 1.

\item[R\refstepcounter{reqnum}\thereqnum \label{R_piece}:] Generate pieces by calling cutting function in game engine. Then get gravity center of each piece.

\item[R\refstepcounter{reqnum}\thereqnum \label{R_Calculate}:]
  Calculate the angle between $v_{0}$ and horizontal($\theta _{1}$). Calculate the angle between x axiom and projection on horizontal of initial speed($\theta _{2}$).

\item[R\refstepcounter{reqnum}\thereqnum \label{R_Output1}:] 
  Calculate and output trace of motion for each piece in the air($S_{x},S_{y},S_{z}$).
  
\item[R\refstepcounter{reqnum}\thereqnum \label{R_Output2}:] 
  Calculate and output trace of motion for each piece on the ground($S_{x},S_{y},S_{z}$).

\end{itemize}

\wss{Functional requirements should reference the instance models.}

\subsection{Nonfunctional Requirements}

Performance is influenced by amount of pieces, capability of GPU and CPU. Other nonfunctional requirements are correctness and reusability.

\wss{More information on the NFRs would improve this document.}

\section{Likely Changes}    

\noindent \begin{itemize}

\item[LC\refstepcounter{lcnum}\thelcnum\label{LC_SOP}:] \aref{A_ground} - In real situation, users may have kinds of terrians and textures. As a result, coefficient of friction needs to be changed correspondingly. 

\end{itemize}

\section{Traceability Matrices and Graphs}

The purpose of the traceability matrices is to provide easy references on what
has to be additionally modified if a certain component is changed.  Every time a
component is changed, the items in the column of that component that are marked
with an ``X'' may have to be modified as well.  Table~\ref{Table:trace} shows the
dependencies of theoretical models, general definitions, data definitions, and
instance models with each other. Table~\ref{Table:R_trace} shows the
dependencies of instance models, requirements, and data constraints on each
other. Table~\ref{Table:A_trace} shows the dependencies of theoretical models,
general definitions, data definitions, instance models, and likely changes on
the assumptions.


\afterpage{
\begin{landscape}
\begin{table}[h!]
\centering
\begin{tabular}{|c|c|c|c|c|c|c|c|}
\hline
	& \aref{A_mechanical}& \aref{A_airFriction}& \aref{A_powder}& \aref{A_initialPE}& \aref{A_ground}& \aref{A_piece}& \aref{A_collision}\\
\hline
\tref{T_CME}        & X& & & & & &\\ \hline
\tref{T_WKF}        & & & & & X& &\\ \hline
\ddref{DD_UAM}        & & & & & X& &\\ \hline
\ddref{DD_Fk}        & & & & & X& &\\ \hline
\iref{IM_angle}        & & & & & & &\\ \hline
\iref{IM_DIA}        & & X& & & & &\\ \hline
\iref{IM_DOG}        & & & & & X& &\\ \hline
\lcref{LC_SOP}        & & X& & & & X&\\ \hline
\end{tabular}
\caption{Traceability Matrix Showing the Connections Between Assumptions and Other Items}
\label{Table:A_trace}
\end{table}
\end{landscape}
}

\begin{table}[h!]
\centering
\begin{tabular}{|c|c|c|c|c|c|c|c|}
\hline        
	& \tref{T_CME}& \tref{T_WKF}& \ddref{DD_UAM}& \ddref{DD_Fk} &\iref{IM_angle} & \iref{IM_DIA}& \iref{IM_DOG} \\
\hline
\tref{T_CME}        & & & & & & X&\\ \hline
\tref{T_WKF}        & & & & X& & &X\\ \hline
\ddref{DD_UAM}        & & & & X& & &X\\ \hline
\ddref{DD_Fk}        & & & & & & &\\ \hline
\iref{IM_angle}        & & & & & & &\\ \hline
\iref{IM_DIA}        & & & & & & &\\ \hline
\iref{IM_DOG}        & & & X& X& & &\\ \hline
\end{tabular}
\caption{Traceability Matrix Showing the Connections Between Items of Different Sections}
\label{Table:trace}
\end{table}

\begin{table}[h!]
\centering
\begin{tabular}{|c|c|c|c|c|c|c|c|c|c|}
\hline
	& \iref{IM_angle}& \iref{IM_DIA}& \iref{IM_DOG}& \rref{R_Inputs}& \rref{R_OutputInputs}& \rref{R_VerifyOutput}& \rref{R_Calculate}& \rref{R_Output1}& \rref{R_Output2} \\
\hline
\iref{IM_angle}            & & & & X& & & X& &\\ \hline
\iref{IM_DIA}            & X& & & & & & X& X&\\ \hline
\iref{IM_DOG}          & & & & X& & & & &X\\ \hline
\rref{R_Inputs}          & & & & & X& & & X&X\\ \hline
\rref{R_OutputInputs}     & & & & & & & & &\\ \hline
\rref{R_VerifyOutput}    & & & & & & & & &\\ \hline
\rref{R_Calculate}   & & & & & & & & X&\\ \hline
\rref{R_Output1}  & & & & & & & & &\\ \hline
\rref{R_Output2}  & & & & & & & & &\\ \hline
\end{tabular}
\caption{Traceability Matrix Showing the Connections Between Requirements and Instance Models}
\label{Table:R_trace}
\end{table}

%The purpose of the traceability graphs is also to provide easy references on
%what has to be additionally modified if a certain component is changed.  The
%arrows in the graphs represent dependencies. The component at the tail of an
%arrow is depended on by the component at the head of that arrow. Therefore, if a
%component is changed, the components that it points to should also be
%changed. Figure~\ref{Fig_ATrace} shows the dependencies of theoretical models,
%general definitions, data definitions, instance models, likely changes, and
%assumptions on each other. Figure~\ref{Fig_RTrace} shows the dependencies of
%instance models, requirements, and data constraints on each other.

% \begin{figure}[h!]
% 	\begin{center}
% 		%\rotatebox{-90}
% 		{
% 			\includegraphics[width=\textwidth]{ATrace.png}
% 		}
% 		\caption{\label{Fig_ATrace} Traceability Matrix Showing the Connections Between Items of Different Sections}
% 	\end{center}
% \end{figure}


% \begin{figure}[h!]
% 	\begin{center}
% 		%\rotatebox{-90}
% 		{
% 			\includegraphics[width=0.7\textwidth]{RTrace.png}
% 		}
% 		\caption{\label{Fig_RTrace} Traceability Matrix Showing the Connections Between Requirements, Instance Models, and Data Constraints}
% 	\end{center}
% \end{figure}

\newpage

\bibliographystyle {plainnat}
\bibliography {../../ReferenceMaterial/References}

\newpage

\section{Appendix}


\subsection{Symbolic Parameters}


\end{document}